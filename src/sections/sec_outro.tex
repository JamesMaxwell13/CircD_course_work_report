\sectionCenteredToc{Заключение}
\label{sec:outro}

В ходе проекта были разработаны и реализованы схемотехнические решения для лазерной гравировки и числового 
программного управления (ЧПУ). Были созданы методы преобразования уровней сигналов, обеспечивающие согласование 
различных напряжений, а также разработаны схемы управления шаговыми двигателями и лазером. 
Эти решения позволяют точно и эффективно управлять процессом гравировки, что упрощает и ускоряет процесс 
создания высококачественных изделий.

Подход, разработанный в рамках данного проекта, имеет потенциал в сравнении с уже существующими решениями на рынке 
лазерных гравировальных станков. Такие технические аспекты важны для современной индустрии производства и 
прототипирования, где лазерная гравировка играет значительную роль.

В завершение, можно отметить, что данный проект демонстрирует важность исследований в области схемотехники и 
автоматизации. Наши усилия направлены на создание решений, которые способствуют более эффективной и доступной 
работе ЧПУ станков, что является большим вкладом в развитие данной отрасли.